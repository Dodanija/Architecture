\documentclass{article}
\usepackage{graphicx} % Required for inserting images

\usepackage[lithuanian]{babel}  
\usepackage[T1]{fontenc}       
\usepackage[utf8]{inputenc}     
\title{Architektūrų AN/UYK-43 ir Intel i960 techninė ataskaita}
\author{VU MIF Informatikos 2 kurso studentas Dovydas Čepas}
\date{Gruodis 2024}
\begin{document}

\maketitle
\newpage
\section{Architektūrų palyginimas}
\subsection{Elementinė bazė ir fizinės savybės}
\subsubsection{AN/UYK-43}
AN/UYK-43 buvo sudarytas iš integrinių grandynų (IC) ir didelio integracijos masto (LSI). Saugojimo atminties technologijoje buvo naudojami puslaidininkiai (VIP Club MN, n.d.). Svoris: 667 kg. - 757 kg. Dydis: 1.83 m. x 50.3 cm. x 56.7 cm. Energijos suvartojimas: 5.5 kW (aušinant oru), 4.7 kW (aušinant vandeniu) (Defense Technical Information Center, 1987).

\subsubsection{Intel i960}
Intel i960 buvo sudarytas iš tranzistorių ir buvo labai didelio integracijos masto (VLSI) (Wikipedia, 2024). Svoris: 24 g (keramikinis) (CPU-World Forum, n.d.). Dydis: 3.5 cm. x 3.5 cm. x 1.54 mm (Intel, n.d.). Energijos suvartojimas: 3.3 W - 5 W (Wikipedia, 2024).

\subsection{Architektūra}
\subsubsection{AN/UYK-43}
Kompiuterio CPU yra bendros paskirties mikroprogramuojamas valdiklis (MPC), kuris vykdo AN/UYK-43 instrukcijų rinkinio architektūrą (ISA) (HP-9020C/AN/UYK-43 Study, 1987). Todėl ji yra CISC pagrindo, o tai yra registrinė.
\subsubsection{Intel i960}
Architektūra yra registrinio pagrindo RISC (Wikipedia, 2024)
\subsection{Adresų mašinos tipas}
\subsubsection{AN/UYK-43}
AN/UYK-43 yra dviejų adresų mašina (pvz: SET A32S15 TO LN(A32S10) \$) (CMS-2Y Programmer's Reference Manual for the AN/UYK-7 and AN/UYK-43.)
\subsubsection{Intel i960}
Intel i960 yra trijų adresų mašina (pvz: addo g2, g3, g4) (ManualsLib, n.d.)
\subsection{Registrai}
\subsection{Požymių bitai}
\subsection{Duomenų plotis (Mašininis žodis)}
\subsection{Atminties išdėstymas}
\subsection{Virtuali atmintis}
\subsection{Komandų sistema (ISA)}
\subsection{Adresavimo būdai}
\subsection{I/O galimybės}
\subsection{Pertraukimai}
\subsection{Duomenų tipai}
\subsection{Greitaveika}
\subsection{Spartinančioji atmintis}
\subsection{Architektūros taikymo sritys}
\subsection{Programinė įranga}
\subsection{Emuliatoriai}
\section{References}
\begin{itemize}
\item CPU-World Forum, n.d. \textit{Intel i960 processor discussion.} Available at: \url{https://www.cpu-world.com/forum/viewtopic.php?p=74549} [Accessed 15 December 2024].
\item Defense Technical Information Center, 1987. \textit{AN/UYK-43 Computer System Specifications.} Available at: \url{https://apps.dtic.mil/sti/tr/pdf/ADA188056.pdf} [Accessed 16 December 2024].

\item Landwehr, C.E., Tretick, R., et al., 1987. \textit{1987 Landwehr Tretick Study.} Available at: \url{http://www.landwehr.org/1987-landwehr-tretick-etal.pdf} [Accessed 15 December 2024].

\item Intel, n.d. \textit{Intel i960 Product Specifications.} Available at: \url{https://community.intel.com/cipcp26785/attachments/cipcp26785/processors/47453/1/INTEL-27300103.PDF} [Accessed 15 December 2024].

\item Mouser Electronics, n.d. \textit{Intel Corporation i960 Linecards.} Available at: \url{https://www.mouser.com/catalog/specsheets/intel%20corporation_i960_linecards.pdf} [Accessed 15 December 2024].

\item ManualsLib, n.d. \textit{Intel i960 Manual.} Available at: \url{https://www.manualslib.com/manual/1315143/Intel-I960.html} [Accessed 16 December 2024].

\item Archive.org, 1986. \textit{CMS-2Y Programmer's Reference Manual for the AN/UYK-7 and AN/UYK-43.} Available at: \url{https://ia902907.us.archive.org/11/items/bitsavers_univacmilimmersReferenceManualfortheANUYK7andANUYK_23389579/CMS-2Y_Programmers_Reference_Manual_for_the_AN_UYK-7_and_AN_UYK-43_Oct86.pdf} [Accessed 16 December 2024].

\item VIP Club MN, n.d. \textit{AN/UYK-43 Historical Overview.} Available at: \url{https://vipclubmn.org/cp32bit.html} [Accessed 15 December 2024].

\item Wikipedia, 2024. \textit{Intel i960.} Available at: \url{https://en.wikipedia.org/wiki/Intel_i960} [Accessed 15 December 2024].

\item HP-9020C/AN/UYK-43 Study, 1987. \textit{System Architecture Analysis.} \url{https://apps.dtic.mil/sti/tr/pdf/ADA188056.pdf} [Accessed 15 December 2024].

\end{itemize}

\end{document}