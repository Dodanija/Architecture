\documentclass{article}
\usepackage{graphicx} % Required for inserting images

\usepackage[lithuanian]{babel}  
\usepackage[T1]{fontenc}       
\usepackage[utf8]{inputenc}     
\title{Architektūrų AN/UYK-43 ir Intel i960 techninė ataskaita}
\author{VU MIF Informatikos 2 kurso studentas Dovydas Čepas}
\date{Gruodis 2024}
\begin{document}

\maketitle
\newpage
\section{Architektūrų palyginimas}
\subsection{Elementinė bazė ir fizinės savybės}
\subsubsection{AN/UYK-43}
AN/UYK-43 buvo sudarytas iš integrinių grandynų (IC) ir didelio integracijos masto (LSI). Saugojimo atminties technologijoje buvo naudojami puslaidininkiai (VIP Club MN, 2024.). Svoris: 667 kg. - 757 kg. Dydis: 1.83 m. x 50.3 cm. x 56.7 cm. Energijos suvartojimas: 5.5 kW (aušinant oru), 4.7 kW (aušinant vandeniu) (Defense Technical Information Center, 1987).

\subsubsection{Intel i960}
Intel i960 buvo sudarytas iš tranzistorių ir buvo labai didelio integracijos masto (VLSI) (Wikipedia, 2024). Svoris: 24 g (keramikinis) (CPU-World Forum, n.d.). Dydis: 3.5 cm. x 3.5 cm. x 1.54 mm (Intel, n.d.). Energijos suvartojimas: 3.3 W - 5 W (Wikipedia, 2024).

\subsection{Architektūra}
\subsubsection{AN/UYK-43}
Kompiuterio CPU yra bendros paskirties mikroprogramuojamas valdiklis (MPC), kuris vykdo AN/UYK-43 instrukcijų rinkinio architektūrą (ISA) (HP-9020C/AN/UYK-43 Study, 1987). Todėl ji yra CISC pagrindo.
\subsubsection{Intel i960}
Architektūra yra registrinio pagrindo RISC (Wikipedia, 2024)
\subsection{Adresų mašinos tipas}
\subsubsection{AN/UYK-43}
AN/UYK-43 yra dviejų adresų mašina (pvz: SET A32S15 TO LN(A32S10) \$) (Fleet combat direction systems support activity code 8, 1987)
\subsubsection{Intel i960}
Intel i960 yra trijų adresų mašina (pvz: addo g2, g3, g4) (ManualsLib, n.d.)
\subsection{Registrai}
\subsubsection{Intel i960}
Turi 32 bendrosios paskirties registrus - 16 globalių registrų ir 16 vietinių registrų. Registrų plotis - 32 bitai (Intel, n.d.). Specializuotos paskirties: 4 80 bitų "floating point" registrai (tiksliasniems skaičiams po kablelio) ir kontrolės registrai (interupt'ų valdymui) (Intel Corporation, n.d.).
\subsubsection{AN/UYK-43}
Bendrosios paskirties registrai: A registrai (kaupikliai): 8 registrai, 32 bitų pločio. Naudojami aritmetiniams ir loginėms operacijoms.
B registrai (indeksavimo registrai): 7 registrai, 16 bitų pločio. Naudojami adresų skaičiavimams ir indeksavimui. Specializuoti registrai: P registras (programos skaitiklis): 16 bitų, naudojamas instrukcijų adresams skaičiuoti. S registrai (segmentų registrai): 8 registrai, 32 bitų pločio. Naudojami atminties segmentų valdymui ir adresų generavimui.
Steko rodyklės registrai: 8 registrai, 16 bitų pločio. Naudojami steko valdymui. Atminties apsaugos registrai (SPR): 8 registrai. Apibrėžia atminties prieigos ribas ir režimus. Segmentų identifikacijos registrai (SIR): Naudojami saugumo ir prieigos kontrolės funkcijoms. Registrų duomenų pločiai: 16 bitų registrai: P registras, B registrai, steko rodyklės registrai. 32 bitų registrai: A registrai, S registrai, atminties apsaugos registrai (SPR). Instrukcijų valdymas: P registras seka programos vykdymo eigą. (Landwehr, C.E., Tretick, R., et al., 1987.)
\subsection{Požymių bitai}
https://vipclubmn.org/BitsBakUp/CMS-2Y%20Programmers%20Reference%20Manual%20for%20the%20AN_UYK-7%20and%20AN_UYK-43%20(Fleet%20Combat%20Direction%20Systems%20Support%20Activity%20-%20San%20Diego)%20(October%201,%201986).pdf 
pabaigti vėliau
\subsection{Duomenų plotis (Mašininis žodis)}
\subsubsection{Intel i960}
Mašininis žodis - 32 bitai (Intel Corporation, n.d.)
\subsubsection{AN/UYK-43}
Mašininis žodis - 32 bitai (Landwehr, C.E., Tretick, R., et al., 1987.)
\subsection{Atminties išdėstymas}
\subsection{Virtuali atmintis}
\subsubsection{Intel i960}
Neturėjo virtualios atminties (Niekur nėra informacijos, kad architektūra tokią turėtų, todėl pasitikrinti buvo pateikta užklausa Chatgpt) "The Intel i960 microprocessor did not have built-in support for virtual memory in its standard versions. The i960 was designed primarily as an embedded processor, targeting real-time and embedded systems where virtual memory was not typically needed." (OpenAI, 2024).
\subsubsection{AN/UYK-43}
Neturėjo virtualios atminties (Niekur nėra informacijos, kad architektūra tokią turėtų, todėl pasitikrinti buvo pateikta užklausa Chatgpt) "The AN/UYK-43, a ruggedized military computer introduced by the U.S. Navy in the 1980s, did not have virtual memory in the traditional sense used by modern general-purpose computers." (OpenAI, 2024).
\subsection{Komandų sistema (ISA)}
\subsection{Adresavimo būdai}
\subsection{I/O galimybės}
\subsection{Pertraukimai}
\subsection{Duomenų tipai}
\subsection{Greitaveika}
\subsection{Spartinančioji atmintis}
\subsubsection{AN/UYK-43}
AN/UYK-43 naudojo spartinančią atmintį siekiant padidinti našumą (VIP Club MN, 2024). Kompiuterio spartinančioji atmintis turėjo iki 16384 32 bitų žodžių talpą. Ji veikė kaip didelės spartos buferis tarp procesoriaus ir pagrindinės atminties. (HP-9020C/AN/UYK-43 Study, 1987).
\subsubsection{Intel i960}
Visi Intel i960 procesoriaus variantai naudojo instrukcijų (instruction), tačiau ne visi duomenų (data) spartinančiąją atmintį. Pavyzdžiui Intel i960 80960MC neturėjo duomenų spartinančiosios atminties, o 80960CF jau turėjo. Instrukcijų spartinančiosios atminties dydis buvo nuo 0.5 KB. iki 16 KB., o duomenų spartinančioji atmintis nuo 1 KB. iki 8 KB.
\subsection{Architektūros taikymo sritys}
\subsubsection{AN/UYK-43}
AN/UYK-43 buvo standartinis 32 bitų kompiuteris Jungtinių Amerikos valstijų laivyne, kurio panaudojimas buvo kariniuose laivuose ir povandeniniuose laivuose. (Wikipedia, 2024.) Tai yra universalūs kompiuteriai, naudojami taktinių skaitmeninių sistemų ir posistemių veikloje (pvz., vadovavimo ir kontrolės, žvalgybos bei taktinių ginklų sistemose ir posistemėse) (Defense Technical Information Center, 1981). Daugiau konkrečios informacijos negalima pasiekti, nes tai yra karinė įranga.
\subsubsection{Intel i960}
Intel i960 procesoriai buvo naudojami RAID valdikliuose, tokiuose kaip Adaptec AAR-2400A, Brocade Fibre Channel komutatoriuose, lošimo automatuose, tokiuose kaip IGT Stepper S2000 ir Sega Model 2 žaidimų konsolėse, kariniuose ir kosminiuose įrenginiuose, įskaitant HAL Tejas naikintuvų radarus ir Indijos kosmoso tyrimų organizacijos (ISRO) paleidimo transporto priemonių kompiuterius, jūrų radarų sistemose (Kelvin Hughes ARPA), HP X-terminaluose, Alcatel-Lucent 1000 ADSL plačiajuosčio ryšio modemuose bei SATA RAID valdikliuose. Pavyzdžiui: "Intel i960 buvo naudojamas Sega Model 2 arkadinių žaidimų sistemoje kaip pagrindinis procesorius, atsakingas už 3D grafikos apdorojimą ir žaidimų logiką. Tai buvo galingas RISC architektūros procesorius, kuris tuo metu pasižymėjo aukštu našumu skaičiavimuose ir įterptinėse sistemose. Dėl šių savybių Intel i960 buvo kritiškai svarbus komponentas, leidęs Sega Model 2 pasiekti didelį žaidimų grafikų ir našumo lygį, kas buvo svarbu populiariems žaidimams, tokiems kaip Virtua Fighter 2, Daytona USA ir Sega Rally Championship." (OpenAI, 2024). Taigi taikymo sritys yra nuo paprastų kompiuterinių žaidimų konsolių iki kosmoso tyrimų ar karinių sistemų.
\subsection{Programinė įranga}
\subsubsection{AN/UYK-43}
Standartinė AN/UYK-43 Executive (SDEX/43) operacinė sistema. SDEX/43 yra
užkoduotas ir prižiūrimas naudojant laivyno standartinę MTASS/L programą. MTASS/L dokumentuotas pagal MIL-STD-1679 reikalavimus. Kitos dvi galimos operacinės sistemos yra RSS ir ATEX. Sistema turi du kalbos
procesorius: CMS-2L kompiliatorių ir MACRO/L assembler'į.(HP-9020C/AN/UYK-43 Study, 1987.) AN/UYK-43 ir susijusiems UYK bei AYK serijos kompiuteriams buvo naudojama standartizuota programavimo kalba CMS-2, sukurta Rand Corporation. (Wikipedia, 2024.) Kita informacija nėra pasiekiama.
\subsubsection{Intel i960}
Intel i960® procesorių šeimą palaiko daugiau nei 40 tiekėjų, siūlančių daugiau nei 100 kūrimo įrankių, tokių kaip įterptiniai emuliatoriai, kompiliatoriai, operacinės sistemos, vertinimo plokštės, surinkėjai, derintojai (gdb960), stebėjimo įrankiai (MON960) ir daugelis kitų. Šie įrankiai padeda sutrumpinti tiek kūrimo ciklą, tiek laiką, reikalingą produktui patekti į rinką. (Intel, 2006.) „CTOOLS960“ ir „GNU/960“, C/C++ kompiliatoriai buvo dalis šių įrankių rinkinių.  Derintojai: „dmp960“, „gdmp960“ – disasembleriai ir objektinių failų iškrovėjai. Profiliuotojai: „gcov960“ – kodo padengimo analizatorius. „ghist960“ – statistinis vykdymo profiliuotojas. „xlate960“ – surinkimo kalbos konverteris. „lnk960“, „gld960“ – jungikliai. Buvo prieinamos šitos bibliotekos: MON960“ – bibliotekos inicializacija statistiniam profiliavimui. „IxWorks“ – naudojama su statistinio profiliavimo įrankiu „ghist960“.(Inter Corporation, n.d.) Tam tikra įranga yra prieinama šiandien (https://www.industry-plaza.com/embedded-software-development-tools-for-p58782.html)
\subsection{Emuliatoriai}
\subsubsection{Intel i960}
Yra emuliatorius: https://i960-emulator.software.informer.com/
\subsubsection{AN/UYK-43}
Emuliatoriaus nėra, nes įranga buvo naudojama JAV karinėje technikoje, todėl ne visa informacija yra viešai prieinama (Wikipedia, 2024).
\section{References}
\begin{itemize}
\item CPU-World Forum, n.d. \textit{Intel i960 processor discussion.} Available at: \url{https://www.cpu-world.com/forum/viewtopic.php?p=74549} [Accessed 15 December 2024].
\item Defense Technical Information Center, 1987. \textit{AN/UYK-43 Computer System Specifications.} Available at: \url{https://apps.dtic.mil/sti/tr/pdf/ADA188056.pdf} [Accessed 16 December 2024].
\item Defense Technical Information Center, 1981. \textit{Issues Concerning the AN/UYK-43 and AN/UYK-44 computer development--
competition and ada transition.} Available at: \url{https://apps.dtic.mil/sti/pdfs/AD1174611.pdf} [Accessed 17 December 2024].
\item Landwehr, C.E., Tretick, R., et al., 1987. \textit{1987 Landwehr Tretick Study.} Available at: \url{http://www.landwehr.org/1987-landwehr-tretick-etal.pdf} [Accessed 15 December 2024].

\item Intel, n.d. \textit{Intel i960 Product Specifications.} Available at: \url{https://community.intel.com/cipcp26785/attachments/cipcp26785/processors/47453/1/INTEL-27300103.PDF} [Accessed 15 December 2024].

\item Intel, 2006. \textit{Intel i960 Development Tools.} Available at: \url{http://developer.intel.com/design/i960/devtools/} [Accessed 17 December 2024].


\item Mouser Electronics, n.d. \textit{Intel Corporation i960 Linecards.} Available at: \url{https://www.mouser.com/catalog/specsheets/intel%20corporation_i960_linecards.pdf} [Accessed 15 December 2024].

\item ManualsLib, n.d. \textit{Intel i960 Manual.} Available at: \url{https://www.manualslib.com/manual/1315143/Intel-I960.html} [Accessed 16 December 2024].

\item Fleet combat direction systems support activity code 8 \textit{CMS-2Y Programmer's Reference Manual for the AN/UYK-7 and AN/UYK-43.} Available at: \url{https://ia902907.us.archive.org/11/items/bitsavers_univacmilimmersReferenceManualfortheANUYK7andANUYK_23389579/CMS-2Y_Programmers_Reference_Manual_for_the_AN_UYK-7_and_AN_UYK-43_Oct86.pdf} [Accessed 16 December 2024].

\item Intel Corporation, n.d. \textit{Intel 80960KB Datasheet.} Available at: \url{https://www.alldatasheet.com/datasheet-pdf/pdf/66071/INTEL/80960KB.html} [Accessed 16 December 2024].

\item Inter Corporation, n.d. \textit{i960 Processor Software Utilities
User’s Guide.} Available at: \url{https://datasheets.chipdb.org/Intel/80960/manuals/48527707.pdf} [Accessed 17 December 2024].


\item VIP Club MN, 2024. \textit{AN/UYK-43 Historical Overview.} Available at: \url{https://vipclubmn.org/cp32bit.html} [Accessed 15 December 2024].

\item Wikipedia, 2024. \textit{Intel i960.} Available at: \url{https://en.wikipedia.org/wiki/Intel_i960} [Accessed 15 December 2024].

\item HP-9020C/AN/UYK-43 Study, 1987. \textit{System Architecture Analysis.} \url{https://apps.dtic.mil/sti/tr/pdf/ADA188056.pdf} [Accessed 15 December 2024].

\item OpenAI, 2024. Response generated using ChatGPT. Question: "Did Intel i960 have virtual memory?" Answer: "The Intel i960 microprocessor did not have built-in support for virtual memory in its standard versions. The i960 was designed primarily as an embedded processor, targeting real-time and embedded systems where virtual memory was not typically needed." [Accessed on: 16 December 2024] [Version: 4o], from: \url{https://chat.openai.com/}

\item OpenAI, 2024. Response generated using ChatGPT. Question: "Did AN/UYK-43 have virtual memory?" Answer: "The AN/UYK-43, a ruggedized military computer introduced by the U.S. Navy in the 1980s, did not have virtual memory in the traditional sense used by modern general-purpose computers." [Accessed on: 16 December 2024] [Version: 4o], from: \url{https://chat.openai.com/}

\item OpenAI, 2024. Response generated using ChatGPT. Question: "What was the use of the Intel i960 in the Sega Model 2 console?" Answer: "The Intel i960 was used in the Sega Model 2 arcade game system as the main processor responsible for 3D graphics processing and game logic. It was a powerful RISC architecture processor that was at the time a high-performance computing and embedded system. Because of these features, the Intel i960 was a critical component that allowed the Sega Model 2 to achieve high levels of gaming graphics and performance, which was important for popular games such as Virtua Fighter 2, Daytona USA, and Sega Rally Championship." [Accessed on: 17 December 2024] [Version: 4o], from: \url{https://chat.openai.com/}

\end{itemize}

\end{document}