\documentclass{article}
\usepackage{graphicx} % Required for inserting images

\usepackage[lithuanian]{babel}  
\usepackage[T1]{fontenc}       
\usepackage[utf8]{inputenc}     
\title{Architektūrų AN/UYK-43 ir Intel i960 techninė ataskaita}
\author{VU MIF Informatikos 2 kurso studentas Dovydas Čepas}
\date{Gruodis 2024}
\begin{document}

\maketitle
\newpage
\section{Architektūrų palyginimas}
\subsection{Elementinė bazė ir fizinės savybės}
\subsubsection{AN/UYK-43}
AN/UYK-43 buvo sudarytas iš integrinių grandynų (IC) ir buvo didelio integracijos masto (LSI). Saugojimo atminties technologijoje buvo naudojami puslaidininkiai. Svoris: 667 kg. - 757 kg. Dydis: 1.83 m. x 50.3 cm. x 56.7 cm. Energijos suvartojimas: 5.5 kw. (aušinant oru), 4.7 kw (aušinant vandeniu)
\subsubsection{Intel i960}
Intel i960 buvo sudarytas iš tranzistorių ir buvo labai didelio integracijos masto (VLSI). Svoris: 24 g. (Keramikinis) Dydis: 3.5 cm. x 3.5 cm. x 1.54 mm. Energijos suvartojimas: 3.3 w. - 5 w.
\subsection{Architektūra}
\subsection{Adresų mašinos tipas}
\subsection{Registrai}
\subsection{Požymių bitai}
\subsection{Duomenų plotis (Mašininis žodis)}
\subsection{Atminties išdėstymas}
\subsection{Virtuali atmintis}
\subsection{Komandų sistema (ISA)}
\subsection{Adresavimo būdai}
\subsection{I/O galimybės}
\subsection{Pertraukimai}
\subsection{Duomenų tipai}
\subsection{Greitaveika}
\subsection{Spartinančioji atmintis}
\subsection{Architektūros taikymo sritys}
\subsection{Programinė įranga}
\subsection{Emuliatoriai}
\section{Citatos}
\subsection{AN/UYK-43}
\verb|https://apps.dtic.mil/sti/tr/pdf/ADA188056.pdf|
\\
\verb|http://www.landwehr.org/1987-landwehr-tretick-etal.pdf|
\\
\verb|https://vipclubmn.org/cp32bit.html|

\subsection{Intel i960}
\verb|https://www.cpu-world.com/forum/viewtopic.php?p=74549|
\\
\verb|https://community.intel.com/cipcp26785/attachments/cipcp26785/processors/47453/1/INTEL-27300103.PDF&ved=2ahUKEwiO74emsaqKAxU_DRAIHT8-FN4QFnoECBYQAQ&usg=AOvVaw1CQaTFnv16oe9AWoo0tmZJ|


\end{document}