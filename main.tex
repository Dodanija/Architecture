\documentclass{article}
\usepackage{graphicx} % Required for inserting images

\usepackage[lithuanian]{babel}  
\usepackage[T1]{fontenc}       
\usepackage[utf8]{inputenc}     
\title{Architektūrų AN/UYK-43 ir Intel i960 techninė ataskaita}
\author{VU MIF Informatikos 2 kurso studentas Dovydas Čepas}
\date{Gruodis 2024}
\begin{document}

\maketitle
\newpage
\section{Architektūrų palyginimas}
\subsection{Elementinė bazė ir fizinės savybės}
\subsubsection{AN/UYK-43}
AN/UYK-43 buvo sudarytas iš integrinių grandynų (IC) ir didelio integracijos masto (LSI). Saugojimo atminties technologijoje buvo naudojami puslaidininkiai (VIP Club, 2024.). Svoris: 667 kg. - 757 kg. Dydis: 1.83 m. x 50.3 cm. x 56.7 cm. Energijos suvartojimas: 5.5 kW (aušinant oru), 4.7 kW (aušinant vandeniu) (Defense Technical Information Center, 1987, p. 14.).

\subsubsection{Intel i960}
Intel i960 buvo sudarytas iš tranzistorių ir buvo labai didelio integracijos masto (VLSI) (Wikipedia, 2024). Svoris: 24 g (keramikinis) (CPU-World Forum, n.d.). Dydis: 3.5 cm. x 3.5 cm. x 1.54 mm (Intel, n.d. p. 30.). Energijos suvartojimas: 3.3 W - 5 W (Wikipedia, 2024.).

\subsection{Architektūra}
\subsubsection{AN/UYK-43}
Kompiuterio CPU yra bendros paskirties mikroprogramuojamas valdiklis (MPC), kuris vykdo AN/UYK-43 instrukcijų rinkinio architektūrą (ISA) (HP-9020C/AN/UYK-43 Study, 1987, p. 22.). AN/UYK-43 naudoja tiesioginio kodo kalbą, gautą iš CMS-2Y asemblerio kalbos. Ši sistema leidžia integruoti mašinos instrukcijas į CMS-2Y programas (Fleet combat direction systems support activity code 8, 1986, p. 462-540)
\subsubsection{Intel i960}
Architektūra yra registrinio pagrindo RISC (Wikipedia, 2024)
\subsection{Adresų mašinos tipas}
\subsubsection{AN/UYK-43}
AN/UYK-43 yra dviejų adresų mašina (pvz: SET A32S15 TO LN(A32S10) \$) (Fleet combat direction systems support activity code 8, 1987, p. 214.)
\subsubsection{Intel i960}
Intel i960 yra trijų adresų mašina (pvz: addo g2, g3, g4) (ManualsLib, n.d. p. 296.)
\subsection{Registrai}
\subsubsection{Intel i960}
Turi 32 bendrosios paskirties registrus - 16 globalių registrų ir 16 vietinių registrų. Registrų plotis - 32 bitai (Intel, n.d, p. 1.). Specializuotos paskirties: 4 80 bitų "floating point" registrai (tiksliasniems skaičiams po kablelio) ir kontrolės registrai (interupt'ų valdymui) (Intel Corporation, n.d, p. 6.).
\subsubsection{AN/UYK-43}
Bendrosios paskirties registrai: A registrai (kaupikliai): 8 registrai, 32 bitų pločio. Naudojami aritmetiniams ir loginėms operacijoms.
B registrai (indeksavimo registrai): 7 registrai, 16 bitų pločio. Naudojami adresų skaičiavimams ir indeksavimui. Specializuoti registrai: P registras (programos skaitiklis): 16 bitų, naudojamas instrukcijų adresams skaičiuoti. S registrai (segmentų registrai): 8 registrai, 32 bitų pločio. Naudojami atminties segmentų valdymui ir adresų generavimui.
Steko rodyklės registrai: 8 registrai, 16 bitų pločio. Naudojami steko valdymui. Atminties apsaugos registrai (SPR): 8 registrai. Apibrėžia atminties prieigos ribas ir režimus. Segmentų identifikacijos registrai (SIR): Naudojami saugumo ir prieigos kontrolės funkcijoms. Registrų duomenų pločiai: 16 bitų registrai: P registras, B registrai, steko rodyklės registrai. 32 bitų registrai: A registrai, S registrai, atminties apsaugos registrai (SPR). Instrukcijų valdymas: P registras seka programos vykdymo eigą. (Landwehr, C.E., Tretick, R., et al., 1987, p. 21 - 22.)
\subsection{Požymių bitai}
AN/UYK-43 architektūroje požymių bitai (flags) buvo naudojamos valdyti programos vykdymo srautą, sistemos konfigūraciją ir kitus veikimo parametrus. Buvo naudojami: FLAGS (LITEMFL - nustato, ar elementas yra konstanta, MULTVFL - nustato, ar elementas turi kelias reikšmes, PTHUNKFL - nustato, ar procedūros pradžia turi būti vykdomas kaip atskira posistemė, naudojant specialią komandą (LBJ B6), ETHUNKFL - veikia taip pat, kaip ir PTHUNKFL, bet taikomas procedūros pabaigai. Visi šitie požymių bitai yra sąlygos kodo požymių bitai), TFLAGS išplėčia FLAGS funkcionalumą ir naudojamas kaip pagalbinės, Cswitch  įjungia arba išjungia specifines funkcijas naudojant sistemines deklaracijas. (Fleet combat direction systems support activity code 8, 1986.)
\subsubsection{Intel i960}
Buvo naudojami požymių bitai (Flags): sąlygos kodo - 010, 001, palyginimo - 001, 010, 100, 000, aritmetinės kontrolės - AC.of, AC.om, AC.nif, pėdsakų (trace) - TC.i, TC.b, TC.c, TC.r, proceso kontrolės - PC.te, PC.em, PC.tfp. (Intel Corporation, n.d.)
\subsection{Duomenų plotis (Mašininis žodis)}
\subsubsection{Intel i960}
Mašininis žodis - 32 bitai (Intel Corporation, n.d.)
\subsubsection{AN/UYK-43}
Mašininis žodis - 32 bitai (Landwehr, C.E., Tretick, R., et al., 1987.)
\subsection{Atminties išdėstymas}
\subsubsection{AN/UYK-43}
Atminties tipas: Puslaidininkiai (SC) ir  magnetinė šerdis (MC) Atminties dydis: MC 32K, 32-bit žodžiai, SC 64K - 512K, 32-bit žodžiai. (Systems Exploration, Inc, 1987.)
\subsubsection{Intel i960}
Ištisinė 32 bitų atminties erdvė, be atminties segmentavimo, išskyrus išplėstą architektūrą, kuri galėjo palaikyti iki 2²²⁶ „objektų“, kiekvienas iki 2³² baitų. (Wikipedia, 2024.) Intel i960 efektyvusis adresų pločio dydis buvo 32 bitai, leidžiantis adresuoti iki 4 GB atminties vienoje plokščioje adresų erdvėje. Tipinis atminties kiekis, naudojamas sistemose, pagrįstose Intel i960, skyrėsi priklausomai nuo taikymo srities, nes procesorius buvo skirtas pramoninėms sistemoms, tinklo įrenginiams ir karinėms/aerokosminėms programoms. Štai suskirstymas pagal naudojimo atvejus: pramoninės sistemos (spausdintuvai, duomenų saugojimo valdikliai, kita pramoninė įranga) - atminties diapazonas: nuo 512 KB iki 4 MB, tinklo įranga (maršrutizatoriai, tinklo sąsajos plokštės) - atminties diapazonas: nuo 2 MB iki 16 MB, karinės ir aerokosminės sistemos - atminties diapazonas: nuo 16 MB iki 64 MB ar daugiau. (OpenAI, 2024.)
\subsection{Virtuali atmintis}
\subsubsection{Intel i960}
Neturėjo virtualios atminties (Niekur nėra informacijos, kad architektūra tokią turėtų, todėl pasitikrinti buvo pateikta užklausa Chatgpt) "The Intel i960 microprocessor did not have built-in support for virtual memory in its standard versions. The i960 was designed primarily as an embedded processor, targeting real-time and embedded systems where virtual memory was not typically needed." (OpenAI, 2024).
\subsubsection{AN/UYK-43}
Neturėjo virtualios atminties (Niekur nėra informacijos, kad architektūra tokią turėtų, todėl pasitikrinti buvo pateikta užklausa Chatgpt) "The AN/UYK-43, a ruggedized military computer introduced by the U.S. Navy in the 1980s, did not have virtual memory in the traditional sense used by modern general-purpose computers." (OpenAI, 2024).
\subsection{Komandų sistema (ISA)}
\subsubsection{AN/UYK-43}
Komandų sistema: AN/UYK-43 naudoja tiesioginio kodo kalbą, gautą iš CMS-2Y asemblerio kalbos. Ši sistema leidžia integruoti mašinos instrukcijas į CMS-2Y programas. Instrukcijos yra suskirstytos į aritmetines ir logines operacijas, duomenų perkėlimo, srauto valdymo, sistemos ir specialiosios. [nenurodyta, kurias palaiko instrukcijas] Palaiko 220 instrukcijų (Fleet combat direction systems support activity code 8, 1986.) AN/UYK-43 kompiuterio instrukcijų rinkinys turi daugiau nei 220 pagrindinių pilno ir pusinio žodžio instrukcijų, kurios teikia operacijas tiesioginiam ir netiesioginiam atminties adresavimui, kintamo ilgio simbolių adresavimui bei privilegijuotam ir neprivilegijuotam vykdymui. (Systems Exploration, Inc, 1987.) Pavyzdžiai:
\item SET A TO B \$
\item MOVE A TO B \$
\item PROCEDURE CHECKIT INPUT V1,V2 
\item OUTPUT V3 \$
\item END-PROC CHECKIT \$
\item FUNCTION TPOS(AZM) A 12 S 5 \$
\item SET ALPHA TO 3+AZM/4 \$
\item IF ALPHA GT 0 THEN RETURN (ALPHA) \$
\item ELSE RETURN (0) \$
\item END-FUNCTION TPOS \$
\subsubsection{Intel i960}
Naudoja RISC komandų sistemą. (Wikipedia, 2024.) Instrukcijos yra suskirstytos į: duomenų judėjimo, aritmetinių (eilinių ir sveikųjų skaičių), loginių, bitų ir bitų laukų, palyginimo, šakojimosi, kvietimo / grįžimo, klaidų, derinimo, procesoriaus valdymo klases. 80960KB procesoriuje išplėčia esamas klases ir prideda naujas grupes: sveikųjų skaičių pavertimas realiaisiais, slankiojo kablelio, sinchroninio judėjimo ir įkrovimo, dešimtainės. (Intel, 1988.) Palaikė šias instrukcijas: register-to-register, immediate, memory access - short, memory access - long, control transfer (Intel 80960KB Datasheet). Turi apie 150 - 200 instrukcijų (Intel, 1988.). Pavyzdžiai: 
\item addo g5, g9, g7
\item subi r3, r5, r6
\item lda 0xfab3, r12
\item ld (r4), g3
\item st g10, (r6)[r7*2]
\item setbit 13, g4, g5
\item muli g5, g6, g7
\item bne g5, g6, target
\subsubsection{Palyginimas}
Skiriasi: AN/UYK-43: Daugiau aukšto lygio programavimo stiliaus, sintaksė labiau panaši į šiuolaikines programavimo kalbas, o
i960 - mažo lygmens, asemblerio stiliaus, naudojant registrus ir tiesiogines atminties nuorodas. AN/UYK-43 reikalauja \$ simbolio eilutės pabaigoje, taip pat naudoja įprastus aritmetikos simbolius, vietoj tekstinių komandų. AN/UYK-43 turi kintamųjų deklaraciją, o i960 dirba su registrais. Panašumai: panašios, srauto instrukcijos, duomenų valdymas.
\subsection{Adresavimo būdai}
AN/UYK-43 palaikė šiuos adresavimo būdus: tiesioginis adresavimas, netiesioginis adresavimas, indeksuotas adresavimas, santykinis adresavimas, adreso poslinkio skaičiavimas. (CMS-2Y Programmer's Reference Manual for the AN/UYK-7 and AN/UYK-43, 1986.) O Intel i960 naudojo šiuos būdus: absoliutus adresavimas, registru netiesioginis adresavimas, registru netiesioginis su poslinkiu, registru netiesioginis su poslinkiu ir indeksu, indeksas su poslinkiu, instrukcijos rodyklės su poslinkiu. (Intel, 1997.) Panašūs buvo: netiesioginis adresavimas, indeksuotas adresavimas, tiesioginis ir absoliutus adresavimas. Skiriasi:  AN/UYK-43 leidžia derinti tiesiogiai koduojamus vardus ir adreso poslinkius, kas nėra aprašyta Intel i960, taip pat i960 leidžia derinti bazinį adresą, indeksą ir poslinkį, o AN/UYK-43 ne.
\subsection{I/O galimybės}
\subsubsection{AN/UYK-43}
UYK-43 vis dar pripažįstamas kompiuterių pramonėje kaip vienas geriausių įrenginių, greitai ir efektyviai priimančių didelius išorinių jutiklių ir kitų duomenų srautus. Po to, kai centriniai procesoriai atlieka reikalingus skaičiavimus ir priima sprendimus, šią informaciją galima greitai perduoti žmogui. Situacijose, kai į laivą nukreipta raketa ar artėja priešo orlaivis, mikrosekundės tampa itin svarbios. Laiku priimti tikslūs ir greiti sprendimai gali išgelbėti gyvybes. (VIP Club, 2024.) Yra  po 1 I/O valdiklį kiekvienam CPU su 32 I/O kanalais. I/O grandinimas (chaining) leidžiamas naudojant I/O valdiklį. (Systems Exploration, Inc, 1987.)
\subsubsection{Intel i960}
80960Rx procesoriai buvo žymimi kaip I/O procesoriai ir turėjo PCI magistralės (2.1 arba 2.2 versijos, priklausomai nuo varianto) įgyvendinimą bei 80960Jx branduolį. Jie galėjo būti naudojami pagrindinėse plokštėse kaip integruoti PCI įrenginiai, taip pat PCI išplėtimo kortose. (Wikipedia, 2024.)
\subsection{Pertraukimai}
AN/UYK-43 uvo naudojami pertraukimai (interrupts). Intel i960 irgi buvo naudojami pertraukimai. Panašumai: abiejose architektūrose buvo naudojami prioritetiniai pertraukimai, leidžiantys valdyti skirtingų svarbos lygių užduotis. Taip pat abi sistemos turėjo nepertraukiamas pertrauktis. Skirtumai: Intel i960 turėjo automatinį konteksto išsaugojimą, o AN/UYK-43 rankinį per registrus išsaugojimą, i960 taip pat turėjo tiesioginę DMA sąsają o AN/UYK-43 neturėjo, i960 taip pat turėjo 32 programuojamus prioritetų lygius, o AN/UYK-43 fiksuotus. (Landwehr, C.E., Tretick, R., et al., 1987), (Intel Corporation, n.d.)
\subsection{Duomenų tipai}
\subsubsection{AN/UYK-43}
Palaikė šiuos duomenų tipus: sveikasis skaičius, fiksuoto kablelio, slankiojo kablelio, simbolių tipas, loginis tipas (boolean). Ir egzotinius: statuso tipas (duomenys su baigtiniu reikšmių skaičiumi, neturinčiu jokios matematinės reikšmės), universalus tipas (bitų seka). Fiksuoto kablelio aritmetika buvo palaikoma naudojant nustatytą bitų skaičių, leidžiantį tiksliai atvaizduoti baigtines trupmenas. Slankiojo kablelio aritmetika buvo įgyvendinta su viengubo ir dvigubo tikslumo duomenų tipais, kurių ilgis buvo atitinkamai 32 ir 64 bitai. (Fleet combat direction systems support activity code 8, 1986.)
\subsubsection{Intel i960}
Palaikė šiuos duomenų tipus: bitas, bitų laukai, sveikasis skaičius (8, 16, 32, 64 bitų), eilinių skaičius (8, 16, 32, 64 bitų sveikasis skaičius be ženklo), trigubas žodis (96 bitai), keturgubas žodis (128 bitai). Procesorius palaiko fiksuoto kablelio aritmetiką naudodamas dvejeto papildinio kodavimą sveikiesiems skaičiams. (Intel, n.d.) Intel i960 mikroprocesorius palaikė tiek fiksuotojo, tiek slankiojo kablelio aritmetiką, tačiau tai priklausė nuo konkretaus i960 šeimos modelio. Fiksuotojo kablelio aritmetika visada buvo palaikoma aparatine įranga, o slankiojo kablelio aritmetika palaikoma aparatine įranga modeliuose su integruotu FPU (slankiojo kablelio bloku). Modeliuose be FPU slankiojo kablelio operacijos buvo atliekamos programinės emuliacijos būdu. (OpenAI, 2024.)
\subsection{Greitaveika}
\subsubsection{AN/UYK-43}
Informacijos internete rasti nepavyko. Specifinės techninės detalės, tokios kaip taktinių generatorių dažniai, ciklų skaičius vienai komandai (CPI) ir tikslūs komandų vykdymo greičiai, nėra plačiai dokumentuotos viešai prieinamuose šaltiniuose. (OpenAI, 2024.)
\subsubsection{Intel i960}
Procesoriaus taktinis dažnis: 10 MHz iki 100 MHz. (Wikipedia, 2024.) Intel i960 (80960JF versija) apdorojoma daugumą komandų su vieno vykdymo ciklu. (Intel, n.d.) i960 80960KB procesorius užtikrina vidutinį komandų vykdymo greitį 7,5 milijono komandų per sekundę (7,5 MIPS) esant 20 MHz taktiniam dažniui. 10 milijonų komandų per sekundę (10 MIPS) esant 25 MHz taktiniam dažniui. (Intel, 1988.)
\subsection{Spartinančioji atmintis}
\subsubsection{AN/UYK-43}
AN/UYK-43 naudojo spartinančią atmintį siekiant padidinti našumą (VIP Club, 2024.). Kompiuterio spartinančioji atmintis turėjo iki 16384 32 bitų žodžių talpą. Ji veikė kaip didelės spartos buferis tarp procesoriaus ir pagrindinės atminties. (Systems Exploration, Inc, 1987.).
\subsubsection{Intel i960}
Visi Intel i960 procesoriaus variantai naudojo instrukcijų (instruction), tačiau ne visi duomenų (data) spartinančiąją atmintį. Pavyzdžiui Intel i960 80960MC neturėjo duomenų spartinančiosios atminties, o 80960CF jau turėjo. Instrukcijų spartinančiosios atminties dydis buvo nuo 0.5 KB. iki 16 KB., o duomenų spartinančioji atmintis nuo 1 KB. iki 8 KB. (Wikipedia, 2024.)
\subsection{Architektūros taikymo sritys}
\subsubsection{AN/UYK-43}
AN/UYK-43 buvo standartinis 32 bitų kompiuteris Jungtinių Amerikos valstijų laivyne, kurio panaudojimas buvo kariniuose laivuose ir povandeniniuose laivuose. (Wikipedia, 2024.) Tai yra universalūs kompiuteriai, naudojami taktinių skaitmeninių sistemų ir posistemių veikloje (pvz., vadovavimo ir kontrolės, žvalgybos bei taktinių ginklų sistemose ir posistemėse) (Defense Technical Information Center, 1981). Daugiau konkrečios informacijos negalima pasiekti, nes tai yra karinė įranga.
\subsubsection{Intel i960}
Intel i960 procesoriai buvo naudojami RAID valdikliuose, tokiuose kaip Adaptec AAR-2400A, Brocade Fibre Channel komutatoriuose, lošimo automatuose, tokiuose kaip IGT Stepper S2000 ir Sega Model 2 žaidimų konsolėse, kariniuose ir kosminiuose įrenginiuose, įskaitant HAL Tejas naikintuvų radarus ir Indijos kosmoso tyrimų organizacijos (ISRO) paleidimo transporto priemonių kompiuterius, jūrų radarų sistemose (Kelvin Hughes ARPA), HP X-terminaluose, Alcatel-Lucent 1000 ADSL plačiajuosčio ryšio modemuose bei SATA RAID valdikliuose. Pavyzdžiui: "Intel i960 buvo naudojamas Sega Model 2 arkadinių žaidimų sistemoje kaip pagrindinis procesorius, atsakingas už 3D grafikos apdorojimą ir žaidimų logiką. Tai buvo galingas RISC architektūros procesorius, kuris tuo metu pasižymėjo aukštu našumu skaičiavimuose ir įterptinėse sistemose. Dėl šių savybių Intel i960 buvo kritiškai svarbus komponentas, leidęs Sega Model 2 pasiekti didelį žaidimų grafikų ir našumo lygį, kas buvo svarbu populiariems žaidimams, tokiems kaip Virtua Fighter 2, Daytona USA ir Sega Rally Championship." (OpenAI, 2024). Taigi taikymo sritys yra nuo paprastų kompiuterinių žaidimų konsolių iki kosmoso tyrimų ar karinių sistemų.
\subsection{Programinė įranga}
\subsubsection{AN/UYK-43}
Standartinė AN/UYK-43 Executive (SDEX/43) operacinė sistema. SDEX/43 yra
užkoduotas ir prižiūrimas naudojant laivyno standartinę MTASS/L programą. MTASS/L dokumentuotas pagal MIL-STD-1679 reikalavimus. Kitos dvi galimos operacinės sistemos yra RSS ir ATEX. Sistema turi du kalbos
procesorius: CMS-2L kompiliatorių ir MACRO/L assembler'į.(Systems Exploration, Inc, 1987.) AN/UYK-43 ir susijusiems UYK bei AYK serijos kompiuteriams buvo naudojama standartizuota programavimo kalba CMS-2, sukurta Rand Corporation. (Wikipedia, 2024.) Kita informacija nėra pasiekiama.
\subsubsection{Intel i960}
Intel i960 procesorių šeimą palaiko daugiau nei 40 tiekėjų, siūlančių daugiau nei 100 kūrimo įrankių, tokių kaip įterptiniai emuliatoriai, kompiliatoriai, operacinės sistemos, vertinimo plokštės, surinkėjai, derintojai (gdb960), stebėjimo įrankiai (MON960) ir daugelis kitų. Šie įrankiai padeda sutrumpinti tiek kūrimo ciklą, tiek laiką, reikalingą produktui patekti į rinką. (Intel, 2006.) „CTOOLS960“ ir „GNU/960“, C/C++ kompiliatoriai buvo dalis šių įrankių rinkinių.  Derintojai: „dmp960“, „gdmp960“ – disasembleriai ir objektinių failų iškrovėjai. Profiliuotojai: „gcov960“ – kodo padengimo analizatorius. „ghist960“ – statistinis vykdymo profiliuotojas. „xlate960“ – surinkimo kalbos konverteris. „lnk960“, „gld960“ – jungikliai. Buvo prieinamos šitos bibliotekos: MON960“ – bibliotekos inicializacija statistiniam profiliavimui. „IxWorks“ – naudojama su statistinio profiliavimo įrankiu „ghist960“.(Inter Corporation, n.d.) Tam tikra įranga yra prieinama šiandien (https://www.industry-plaza.com/embedded-software-development-tools-for-p58782.html)
\subsection{Emuliatoriai}
\subsubsection{Intel i960}
Yra emuliatorius: https://i960-emulator.software.informer.com/
\subsubsection{AN/UYK-43}
Emuliatoriaus nėra, nes įranga buvo naudojama JAV karinėje technikoje, todėl ne visa informacija yra viešai prieinama (Wikipedia, 2024).
\section{References}
\begin{itemize}

\item Intel, 1988. \textit{Intel 80960KB Programmer's Reference Manual.} Available at: \url{http://bitsavers.informatik.uni-stuttgart.de/components/intel/i960/80960KB_Programmers_Reference_Manual_Mar88.pdf} [Accessed 16 December 2024].

\item CPU-World Forum, n.d. \textit{Intel i960 processor discussion.} Available at: \url{https://www.cpu-world.com/forum/viewtopic.php?p=74549} [Accessed 15 December 2024].

\item Defense Technical Information Center, 1987. \textit{AN/UYK-43 Computer System Specifications.} Available at: \url{https://apps.dtic.mil/sti/tr/pdf/ADA188056.pdf} [Accessed 16 December 2024].

\item Defense Technical Information Center, 1981. \textit{Issues Concerning the AN/UYK-43 and AN/UYK-44 computer development--competition and ada transition.} Available at: \url{https://apps.dtic.mil/sti/pdfs/AD1174611.pdf} [Accessed 17 December 2024].

\item Landwehr, C.E., Tretick, R., et al., 1987. \textit{1987 Landwehr Tretick Study.} Available at: \url{http://www.landwehr.org/1987-landwehr-tretick-etal.pdf} [Accessed 15 December 2024].

\item Intel, n.d. \textit{Intel i960 Product Specifications.} Available at: \url{https://community.intel.com/cipcp26785/attachments/cipcp26785/processors/47453/1/INTEL-27300103.PDF} [Accessed 15 December 2024].

\item Intel, 2006. \textit{Intel i960 Development Tools.} Available at: \url{http://developer.intel.com/design/i960/devtools/} [Accessed 17 December 2024].


\item Intel, 2002. \textit{Intel Corporation i960 Linecards.} Available at: \url{https://www.mouser.com/catalog/specsheets/intel%20corporation_i960_linecards.pdf} [Accessed 15 December 2024].

\item ManualsLib, n.d. \textit{Intel i960 Manual.} Available at: \url{https://www.manualslib.com/manual/1315143/Intel-I960.html} [Accessed 16 December 2024].

\item Fleet combat direction systems support activity code 8, 1986. \textit{CMS-2Y Programmer's Reference Manual for the AN/UYK-7 and AN/UYK-43, 1986.} Available at: \url{https://ia902907.us.archive.org/11/items/bitsavers_univacmilimmersReferenceManualfortheANUYK7andANUYK_23389579/CMS-2Y_Programmers_Reference_Manual_for_the_AN_UYK-7_and_AN_UYK-43_Oct86.pdf} [Accessed 16 December 2024].

\item Intel Corporation, n.d. \textit{Intel 80960KB Datasheet.} Available at: \url{https://www.alldatasheet.com/datasheet-pdf/pdf/66071/INTEL/80960KB.html} [Accessed 16 December 2024].

\item Intel, 1997. \textit{i960 Rx I/O Microprocessor Developer’s Manual.} Available at: \url{https://datasheets.chipdb.org/Intel/80960/manuals/27273602.PDF} [Accessed 18 December 2024].

\item Inter Corporation, n.d. \textit{i960 Processor Software Utilities
User’s Guide.} Available at: \url{https://datasheets.chipdb.org/Intel/80960/manuals/48527707.pdf} [Accessed 17 December 2024].


\item VIP Club, 2024. \textit{32-bit Computers, Chapter 52.} Available at: \url{https://vipclubmn.org/cp32bit.html} [Accessed 15 December 2024].

\item Wikipedia, 2024. \textit{Intel i960.} Available at: \url{https://en.wikipedia.org/wiki/Intel_i960} [Accessed 15 December 2024].

\item Wikipedia, 2024. \textit{Intel i960.} Available at: \url{https://en.wikipedia.org/wiki/AN/UYK-43} [Accessed 15 December 2024].

\item Systems Exploration, Inc, 1987. \textit{HP-9020C/AN/UYK-43 Study} \url{https://apps.dtic.mil/sti/tr/pdf/ADA188056.pdf} [Accessed 15 December 2024].

\item OpenAI, 2024. Response generated using ChatGPT. Question: "Did Intel i960 have virtual memory?" Answer: "The Intel i960 microprocessor did not have built-in support for virtual memory in its standard versions. The i960 was designed primarily as an embedded processor, targeting real-time and embedded systems where virtual memory was not typically needed." [Accessed on: 16 December 2024] [Version: 4o], from: \url{https://chat.openai.com/}

\item OpenAI, 2024. Response generated using ChatGPT. Question: "Did AN/UYK-43 have virtual memory?" Answer: "The AN/UYK-43, a ruggedized military computer introduced by the U.S. Navy in the 1980s, did not have virtual memory in the traditional sense used by modern general-purpose computers." [Accessed on: 16 December 2024] [Version: 4o], from: \url{https://chat.openai.com/}

\item OpenAI, 2024. Response generated using ChatGPT. Question: "What was the use of the Intel i960 in the Sega Model 2 console?" Answer: "The Intel i960 was used in the Sega Model 2 arcade game system as the main processor responsible for 3D graphics processing and game logic. It was a powerful RISC architecture processor that was at the time a high-performance computing and embedded system. Because of these features, the Intel i960 was a critical component that allowed the Sega Model 2 to achieve high levels of gaming graphics and performance, which was important for popular games such as Virtua Fighter 2, Daytona USA, and Sega Rally Championship." [Accessed on: 17 December 2024] [Version: 4o], from: \url{https://chat.openai.com/}

\item OpenAI, 2024. Response generated using ChatGPT. Question: "Intel i960 What was  the (effective) width of the address?" Answer: "The effective address width of the Intel i960 was 32 bits, allowing it to address up to 4 GB of memory in a flat address space" [Accessed on: 18 December 2024] [Version: 4o], from: \url{https://chat.openai.com/}

\item OpenAI, 2024. Response generated using ChatGPT. Question: "Intel i960 What was  the (effective) width of the address?" Answer: "The effective address width of the Intel i960 was 32 bits, allowing it to address up to 4 GB of memory in a flat address space" [Accessed on: 18 December 2024] [Version: 4o], from: \url{https://chat.openai.com/}

\item OpenAI, 2024. Response generated using ChatGPT. Question: "What was the typical amount of memory with which the system was used? Intel i960" Answer: "The typical amount of memory used with systems based on the Intel i960 varied depending on the application, as the processor was designed for embedded systems, networking devices, and military/aerospace applications. Here’s a breakdown of typical cases: Embedded Systems (Printers, Storage Controllers, Industrial Equipment) - Memory Range: 512 KB to 4 MB, networking Equipment (Routers, Network Interface Cards) - memory Range: 2 MB to 16 MB, Military and Aerospace Systems - memory Range: 16 MB to 64 MB or more" [Accessed on: 18 December 2024] [Version: 4o], from: \url{https://chat.openai.com/}

\item OpenAI, 2024. Response generated using ChatGPT. Question: "Intel i960 Was fixed point or floating point supported by hardware and how?" Answer: "The Intel i960 microprocessor supported both fixed-point and floating-point arithmetic, but in different ways depending on the specific model of the i960 family. Fixed-Point Arithmetic: Always supported by hardware. Floating-Point Arithmetic: Supported by hardware in models with an integrated FPU, otherwise handled via software emulation." [Accessed on: 18 December 2024] [Version: 4o], from: \url{https://chat.openai.com/}

\item OpenAI, 2024. Response generated using ChatGPT. Question: "What was the speed of AN/UYK-43 system? What were the clock frequencies, clock cycles per instruction, instruction rates?" Answer: "Specific technical details such as clock frequencies, cycles per instruction (CPI), and exact instruction rates are not extensively documented in publicly available sources" [Accessed on: 18 December 2024] [Version: 4o], from: \url{https://chat.openai.com/}

\end{itemize}

\end{document}