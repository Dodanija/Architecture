\documentclass{article}
\usepackage{graphicx} % Required for inserting images

\usepackage[lithuanian]{babel}  
\usepackage[T1]{fontenc}       
\usepackage[utf8]{inputenc}     
\title{Architektūrų AN/UYK-43 ir Intel i960 techninė ataskaita}
\author{VU MIF Informatikos 2 kurso studentas Dovydas Čepas}
\date{Gruodis 2024}
\begin{document}

\maketitle
\newpage
\section{Architektūrų palyginimas}
\subsection{Elementinė bazė ir fizinės savybės}
\subsubsection{AN/UYK-43}
AN/UYK-43 buvo sudarytas iš integrinių grandynų (IC) ir didelio integracijos masto (LSI). Saugojimo atminties technologijoje buvo naudojami puslaidininkiai (VIP Club MN, n.d.). Svoris: 667 kg. - 757 kg. Dydis: 1.83 m. x 50.3 cm. x 56.7 cm. Energijos suvartojimas: 5.5 kW (aušinant oru), 4.7 kW (aušinant vandeniu) (Defense Technical Information Center, 1987).

\subsubsection{Intel i960}
Intel i960 buvo sudarytas iš tranzistorių ir buvo labai didelio integracijos masto (VLSI) (Wikipedia, 2024). Svoris: 24 g (keramikinis) (CPU-World Forum, n.d.). Dydis: 3.5 cm. x 3.5 cm. x 1.54 mm (Intel, n.d.). Energijos suvartojimas: 3.3 W - 5 W (Wikipedia, 2024).

\subsection{Architektūra}
\subsubsection{AN/UYK-43}
Kompiuterio CPU yra bendros paskirties mikroprogramuojamas valdiklis (MPC), kuris vykdo AN/UYK-43 instrukcijų rinkinio architektūrą (ISA) (HP-9020C/AN/UYK-43 Study, 1987). Todėl ji yra CISC pagrindo, o tai yra registrinė.
\subsubsection{Intel i960}
Architektūra yra registrinio pagrindo RISC (Wikipedia, 2024)
\subsection{Adresų mašinos tipas}
\subsubsection{AN/UYK-43}
AN/UYK-43 yra dviejų adresų mašina (pvz: SET A32S15 TO LN(A32S10) \$) (Fleet combat direction systems support activity code 8, 1987)
\subsubsection{Intel i960}
Intel i960 yra trijų adresų mašina (pvz: addo g2, g3, g4) (ManualsLib, n.d.)
\subsection{Registrai}
\subsubsection{Intel i960}
Turi 32 bendrosios paskirties registrus - 16 globalių registrų ir 16 vietinių registrų. Registrų plotis - 32 bitai (Intel, n.d.). Specializuotos paskirties: 4 80 bitų "floating point" registrai (tiksliasniems skaičiams po kablelio) ir kontrolės registrai (interupt'ų valdymui) (Intel Corporation, n.d.).
\subsubsection{AN/UYK-43}
Bendrosios paskirties registrai: A registrai (kaupikliai): 8 registrai, 32 bitų pločio. Naudojami aritmetiniams ir loginėms operacijoms.
B registrai (indeksavimo registrai): 7 registrai, 16 bitų pločio. Naudojami adresų skaičiavimams ir indeksavimui. Specializuoti registrai: P registras (programos skaitiklis): 16 bitų, naudojamas instrukcijų adresams skaičiuoti. S registrai (segmentų registrai): 8 registrai, 32 bitų pločio. Naudojami atminties segmentų valdymui ir adresų generavimui.
Steko rodyklės registrai: 8 registrai, 16 bitų pločio. Naudojami steko valdymui. Atminties apsaugos registrai (SPR): 8 registrai. Apibrėžia atminties prieigos ribas ir režimus. Segmentų identifikacijos registrai (SIR): Naudojami saugumo ir prieigos kontrolės funkcijoms. Registrų duomenų pločiai: 16 bitų registrai: P registras, B registrai, steko rodyklės registrai. 32 bitų registrai: A registrai, S registrai, atminties apsaugos registrai (SPR). Instrukcijų valdymas: P registras seka programos vykdymo eigą. (Landwehr, C.E., Tretick, R., et al., 1987.)
\subsection{Požymių bitai}
https://vipclubmn.org/BitsBakUp/CMS-2Y%20Programmers%20Reference%20Manual%20for%20the%20AN_UYK-7%20and%20AN_UYK-43%20(Fleet%20Combat%20Direction%20Systems%20Support%20Activity%20-%20San%20Diego)%20(October%201,%201986).pdf 
pabaigti vėliau
\subsection{Duomenų plotis (Mašininis žodis)}
\subsubsection{Intel i960} (Intel Corporation, n.d.)
Mašininis žodis - 32 bitai
\subsubsection{AN/UYK-43}
Mašininis žodis - 32 bitai (Landwehr, C.E., Tretick, R., et al., 1987.)
\subsection{Atminties išdėstymas}
\subsection{Virtuali atmintis}
\subsection{Komandų sistema (ISA)}
\subsection{Adresavimo būdai}
\subsection{I/O galimybės}
\subsection{Pertraukimai}
\subsection{Duomenų tipai}
\subsection{Greitaveika}
\subsection{Spartinančioji atmintis}
\subsection{Architektūros taikymo sritys}
\subsection{Programinė įranga}
\subsection{Emuliatoriai}
\subsubsection{Intel i960}
Yra emuliatorius: https://i960-emulator.software.informer.com/
\subsubsection{AN/UYK-43}
Emuliatoriaus nėra, nes įranga buvo naudojama JAV karinėje technikoje, todėl ne visa informacija yra viešai prieinama (Wikipedia, 2024).
\section{References}
\begin{itemize}
\item CPU-World Forum, n.d. \textit{Intel i960 processor discussion.} Available at: \url{https://www.cpu-world.com/forum/viewtopic.php?p=74549} [Accessed 15 December 2024].
\item Defense Technical Information Center, 1987. \textit{AN/UYK-43 Computer System Specifications.} Available at: \url{https://apps.dtic.mil/sti/tr/pdf/ADA188056.pdf} [Accessed 16 December 2024].

\item Landwehr, C.E., Tretick, R., et al., 1987. \textit{1987 Landwehr Tretick Study.} Available at: \url{http://www.landwehr.org/1987-landwehr-tretick-etal.pdf} [Accessed 15 December 2024].

\item Intel, n.d. \textit{Intel i960 Product Specifications.} Available at: \url{https://community.intel.com/cipcp26785/attachments/cipcp26785/processors/47453/1/INTEL-27300103.PDF} [Accessed 15 December 2024].

\item Mouser Electronics, n.d. \textit{Intel Corporation i960 Linecards.} Available at: \url{https://www.mouser.com/catalog/specsheets/intel%20corporation_i960_linecards.pdf} [Accessed 15 December 2024].

\item ManualsLib, n.d. \textit{Intel i960 Manual.} Available at: \url{https://www.manualslib.com/manual/1315143/Intel-I960.html} [Accessed 16 December 2024].

\item Fleet combat direction systems support activity code 8 \textit{CMS-2Y Programmer's Reference Manual for the AN/UYK-7 and AN/UYK-43.} Available at: \url{https://ia902907.us.archive.org/11/items/bitsavers_univacmilimmersReferenceManualfortheANUYK7andANUYK_23389579/CMS-2Y_Programmers_Reference_Manual_for_the_AN_UYK-7_and_AN_UYK-43_Oct86.pdf} [Accessed 16 December 2024].

\item Intel Corporation, n.d. \textit{Intel 80960KB Datasheet.} Available at: \url{https://www.alldatasheet.com/datasheet-pdf/pdf/66071/INTEL/80960KB.html} [Accessed 16 December 2024].


\item VIP Club MN, n.d. \textit{AN/UYK-43 Historical Overview.} Available at: \url{https://vipclubmn.org/cp32bit.html} [Accessed 15 December 2024].

\item Wikipedia, 2024. \textit{Intel i960.} Available at: \url{https://en.wikipedia.org/wiki/Intel_i960} [Accessed 15 December 2024].

\item HP-9020C/AN/UYK-43 Study, 1987. \textit{System Architecture Analysis.} \url{https://apps.dtic.mil/sti/tr/pdf/ADA188056.pdf} [Accessed 15 December 2024].

\end{itemize}

\end{document}